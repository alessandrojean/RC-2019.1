\question
Aplicação cliente-servidor UDP.

\begin{parts}
  \part
  Execute primeiro o servidor (\verb|UDPServer.java|) e depois o cliente
  (\verb|UDPClient.java|). O que aconteceu? Justifique.

  \begin{solution}
    O cliente é capaz de se comunicar com o servidor através de datagramas.
    Enquanto novas mensagens não são enviados ao servidor, o mesmo bloqueia
    no laço de repetição infinito. Ao receber a mensagem, em formato de 
    \verb|String|, o servidor altera-a passando para letras maiúsculas, 
    criando um bloco de dados que será colocado em outro pacote de datagrama
    para ser enviado de volta ao cliente. O cliente, no entanto, executa
    uma única vez, encerrando-o após o servidor executar o serviço,
    liberando os recursos que foram alocados.

    \begin{Verbatim}[label={\$ java UDPServer}]
    RECEIVED FROM CLIENT 127.0.0.1: Este eh um teste
    \end{Verbatim}

    \begin{Verbatim}[label={\$ java UDPClient}]
    Este eh um teste
    RECEIVED FROM SERVER: ESTE EH UM TESTE
    \end{Verbatim}
  \end{solution}

  \part
  Execute primeiro o cliente e depois o servidor. O que aconteceu?
  Justifique.

  \begin{solution}
    O protocolo de comunicação falha, pois supõe-se que sempre há um servidor 
    do outro lado, o cliente fica bloqueado esperando uma resposta que
    nunca virá. Para haver a comunicação é necessário que ambos
    cliente e servidor estejam trabalhando ao mesmo tempo para haver 
    comunicação no protocolo estabelecido.

    \begin{Verbatim}[label={\$ java UDPClient}]
    Teste
    \end{Verbatim}

    \begin{Verbatim}[label={\$ java UDPServer}]
    [Sem saída]
    \end{Verbatim}
  \end{solution}

  \part
  Altere as portas do servidor e cliente. O que acontece se as portas
  forem diferentes? Justifique.

  \begin{solution}
    A porta é o endereço da camada de transporte, ele que permite que
    o pacote seja endereçado ao processo correto. Não há resposta
    pois o cliente envia para uma porta que não há nenhum servidor
    escutando, fazendo com que o pacote seja perdido. Com isto,
    o cliente fica travado pois nunca terá resposta.

    \begin{Verbatim}[label={\$ java UDPServer}]
    [Sem saída]
    \end{Verbatim}

    \begin{Verbatim}[label={\$ java UDPClient}]
    Este eh um teste
    \end{Verbatim}
  \end{solution}

  \part
  Execute servidor e cliente em máquinas diferentes. Em seguida, envie
  várias mensagens ao servidor a partir de várias máquinas diferentes
  (ao mesmo tempo!). O que aconteceu?

  \begin{solution}
    O servidor recebe as mensagens endereçadas a ele, a partir
    de máquinas diferentes e ao mesmo tempo. Os pacotes são recebidos
    e o servidor envia a resposta.

    \begin{Verbatim}[label={\$ java UDPServer}]
    RECEIVED FROM CLIENT 172.17.13.228: Ola
    \end{Verbatim}

    \begin{Verbatim}[label={\$ java UDPClient}]
    Ola
    RECEIVED FROM SERVER: OLA
    \end{Verbatim}
  \end{solution}

  \part
  Modifique o programa cliente para enviar mais de uma mensagem 
  digitada pelo usuário. É necessário alterar também o código
  do servidor? Justifique.

  \begin{solution}
    Não é necessário alterar também o código do servidor já que inicialmente
    o mesmo já está habilitado para ouvir infinitamente novos pacotes
    que sejam endereçados a ele.

    \begin{Verbatim}[label={\$ java UDPServer}]
    RECEIVED FROM CLIENT 127.0.0.1: Teste1
    RECEIVED FROM CLIENT 127.0.0.1: Teste2
    RECEIVED FROM CLIENT 127.0.0.1: Teste3
    \end{Verbatim}

    \begin{Verbatim}[label={\$ java UDPClientModified}]
    Teste1
    RECEIVED FROM SERVER: TESTE1
    Teste2
    RECEIVED FROM SERVER: TESTE2
    Teste3
    RECEIVED FROM SERVER: TESTE3
    /quit
    \end{Verbatim}
  \end{solution}

  \part
  Modifique o programa para que o servidor não responda a um usuário
  indesejado (\emph{white list} ou \emph{black list}).

  \begin{solution}
    Foi realizado a implementação de uma \emph{black list} no servidor,
    através da criação de uma \verb|List| que contém os endereços IPs
    bloqueados e, durante a recepção de um pacote, verifica-se
    se o IP está na lista dos bloqueados.

    \begin{Verbatim}[label={\$ java UDPServerModified}]
    BLOCKED FROM CLIENT 172.17.13.228: Estou bloqueado
    \end{Verbatim}

    \begin{Verbatim}[label={\$ java UDPClientModified}]
    Estou bloqueado
    \end{Verbatim}
  \end{solution}
\end{parts}

\question
Desenvolvimento de aplicação cliente-servidor UDP.

\begin{parts}
  \part
  Crie um procolo de aplicação qualquer e implemente o cliente e
  servidor para esse protocolo.

  \begin{solution}
    O protocolo estabelecido permite que o cliente envie uma frase
    para o servidor. Este verifica se a frase é palíndroma, retornando
    como resposta \verb|SIM| caso positivo e \verb|NÃO| caso negativo.

    \begin{Verbatim}[label={\$ java UDPServerModifiedChecker}, fontsize=\small]
    RECEIVED FROM CLIENT 172.17.13.228: banana
    RECEIVED FROM CLIENT 172.17.13.228: Socorram me subi no onibus em Marrocos
    \end{Verbatim}

    \begin{Verbatim}[label={\$ java UDPClientModified}]
    banana
    RECEIVED FROM SERVER: NÃO
    Socorram me subi no onibus em Marrocos
    RECEIVED FROM SERVER: SIM
    /quit
    \end{Verbatim}
  \end{solution}
\end{parts}
