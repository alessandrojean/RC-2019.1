\question
Aplicação cliente-servidor TCP.

\begin{parts}
  \part
  Execute primeiro o servidor (\verb|TCPServer.java|) e depois o
  cliente (\verb|TCPClient.java|). O que aconteceu?

  \begin{solution}
    O servidor estabelece um \verb|ServerSocket|, ficando em um loop 
    infinito esperando por uma conexão. Se houver uma conexão, o servidor
    lê as \verb|Strings| enviadas pelo cliente, passando-as para maiúscula
    e devolvendo para o mesmo. A abstração utilizada no \emph{socket}
    é uma de fluxo, onde há um de entrada, pelo teclado, e saída,
    para envio dos dados. A conexão entre servidor-cliente é mantida
    até que o cliente envie uma linha em branco.

    \begin{Verbatim}[label={\$ java TCPServer}]
    Waiting for connection at port 9000.
    Connection established from /127.0.0.1
    Waiting for connection at port 9000.
    \end{Verbatim}

    \begin{Verbatim}[label={\$ java TCPClient}]
    Ola, este eh um teste
    OLA, ESTE EH UM TESTE
    Mensagem para o servidor
    MENSAGEM PARA O SERVIDOR
   
    \end{Verbatim}
  \end{solution}

  \part
  Execute primeiro o cliente e depois o servidor. O que aconteceu?
  Justifique.

  \begin{solution}
    Para o cliente estabelecer uma conexão com o servidor, este deve
    já estar operando. Caso o servidor não esteja operando, a camada
    de transporte do servidor enviará uma mensagem ao cliente informando
    que não é possível estabelecer uma conexão, gerando uma exceção no
    Java, onde a conexão é recusada.

    \begin{Verbatim}[label={\$ java TCPClient}, fontsize=\scriptsize]
    Exception in thread "main" java.net.ConnectException: Conexão recusada (Connection refused)
      at java.net.PlainSocketImpl.socketConnect(Native Method)
      at java.net.AbstractPlainSocketImpl.doConnect(AbstractPlainSocketImpl.java:350)
      at java.net.AbstractPlainSocketImpl.connectToAddress(AbstractPlainSocketImpl.java:206)
      at java.net.AbstractPlainSocketImpl.connect(AbstractPlainSocketImpl.java:188)
      at java.net.SocksSocketImpl.connect(SocksSocketImpl.java:392)
      at java.net.Socket.connect(Socket.java:589)
      at java.net.Socket.connect(Socket.java:538)
      at java.net.Socket.<init>(Socket.java:434)
      at java.net.Socket.<init>(Socket.java:211)
      at TCPClient.main(TCPClient.java:11)
    \end{Verbatim}

    \begin{Verbatim}[label={\$ java TCPServer}]
    Waiting for connection at port 9000.    
    \end{Verbatim}
  \end{solution}

  \part
  Altere as portas do servidor e cliente. O que acontece se as portas
  forem diferentes? Justifique.

  \begin{solution}
    Para se estabelecer uma comunicação, além do número do IP
    é necessário saber a qual processo ela se direciona. Logo,
    como o servidor opera na porta \verb|9000|, o cliente precisa 
    necessariamente utilizar esta porta para estabelecer uma conexão
    e trocar os dados sem nenhum problema.

    \begin{Verbatim}[label={\$ java TCPServer}]
    Waiting for connection at port 9000.
    \end{Verbatim}

    \begin{Verbatim}[label={\$ java TCPClient}, fontsize=\scriptsize]
    Exception in thread "main" java.net.ConnectException: Conexão recusada (Connection refused)
      at java.net.PlainSocketImpl.socketConnect(Native Method)
      at java.net.AbstractPlainSocketImpl.doConnect(AbstractPlainSocketImpl.java:350)
      at java.net.AbstractPlainSocketImpl.connectToAddress(AbstractPlainSocketImpl.java:206)
      at java.net.AbstractPlainSocketImpl.connect(AbstractPlainSocketImpl.java:188)
      at java.net.SocksSocketImpl.connect(SocksSocketImpl.java:392)
      at java.net.Socket.connect(Socket.java:589)
      at java.net.Socket.connect(Socket.java:538)
      at java.net.Socket.<init>(Socket.java:434)
      at java.net.Socket.<init>(Socket.java:211)
      at TCPClient.main(TCPClient.java:11)
    \end{Verbatim}
  \end{solution}

  \part
  Execute servidor e cliente em máquinas diferentes. Em seguida, envie
  várias mensagens ao servidor a partir de várias máquinas diferentes
  (ao mesmo tempo!). O que aconteceu? Justifique.

  \begin{solution}
    Não é possível estabelecer uma conexão com o servidor quando
    um cliente já está conectado ao servidor, já que a comunicação
    é sempre um-para-um, uma relação entre o \emph{host} de origem e o
    \emph{host} de destino. Para permitir que múltiplos clientes
    se conectem ao servidor, será necessário modificar seu código.

    \begin{Verbatim}[label={\$ java TCPServer}]
    Waiting for connection at port 9000.
    Connection established from /172.17.13.228
    Waiting for connection at port 9000
    \end{Verbatim}

    \begin{Verbatim}[label={\$ java TCPClient}]
    Este eh um teste
    ESTE EH UM TESTE
    Enviando de maquina diferente
    ENVIANDO DE MAQUINA DIFERENTE

    \end{Verbatim}
  \end{solution}

  \part
  Verifique o cliente e servidor UDP da aula anterior. Quais as
  principais diferenças para o cliente e servidor TCP?

  \begin{solution}
    No UDP não se estabelece uma conexão entre cliente-servidor.
    O servidor recebe os datagramas e envia esses datagramas.
    Já no TCP é necessário estabelecer uma conexão para a comunicação,
    além do fato que o servidor somente aceita um cliente conectado
    por vez. No UDP também não é necessário encerrar a conexão,
    visto que ela nunca é estabelecida, enquanto no TCP é necessário
    encerrar a conexão para liberar o servidor.
  \end{solution}

  \part
  Modifique o programa cliente para que permaneça conectado enviando
  mensagens ao servidor. A conexão só será desfeita se o cliente enviar
  o comando ``tchau'' para o servidor. Faça também as modificações
  necessárias no código do servidor.

  \begin{solution}
    Modificou-se a verificação no cliente para quando o usuário dava
    um enter para sinalizar que queria encerrar a conexão. Após o usuário
    enviar o comando ``tchau'', este comando é enviado ao servidor e o
    cliente se desconecta, bem como o servidor que identifica o comando
    e se desconecta também.\\

    \begin{Verbatim}[label={\$ java TCPServerModified}]
    Waiting for connection at port 9000.
    Connection established from /127.0.0.1
    Waiting for connection at port 9000.
    \end{Verbatim}

    \begin{Verbatim}[label={\$ java TCPClientModified}]
    Ola, este eh um teste
    OLA, ESTE EH UM TESTE
    Tambem um teste
    TAMBEM UM TESTE
    tchau
    \end{Verbatim}
  \end{solution}
\end{parts}

\question
Desenvolvimento de um servidor TCP com múltiplos clientes.

\begin{parts}
  \part
  Implemente um servidor que atenda a vários clientes simultaneamente.

  \begin{solution}
    O servidor fica em um \emph{loop} infinito aguardando por conexões,
    quando ele recebe alguma, inicia-se uma nova \emph{thread} para
    a conexão, permitindo com que vários clientes sejam atendidos
    ao mesmo tempo.

    \begin{Verbatim}[label={\$ java TCPServerMultiple}]
    Waiting for connection at port 9000.
    Waiting for connection at port 9000.
    Connection established from /172.17.13.228
    Waiting for connection at port 9000.
    Connection established from /172.17.13.220
    \end{Verbatim}

    \begin{Verbatim}[label={\$ java TCPClient [172.17.13.220]}]
    Este eh um teste
    ESTE EH UM TESTE
    Tambem um teste
    TAMBEM UM TESTE

    \end{Verbatim}

    \begin{Verbatim}[label={\$ java TCPClient [172.17.13.228]}]
    Enviando da outra maquina
    ENVIANDO DA OUTRA MAQUINA
    Por enquanto tah tudo tranquilo
    POR ENQUANTO TAH TUDO TRANQUILO

    \end{Verbatim}
  \end{solution}
\end{parts}
